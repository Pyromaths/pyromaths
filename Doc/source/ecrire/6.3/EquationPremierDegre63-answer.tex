\exercice*

\begin{align*}
  (( a )) x (( "%+d"|format(b) )) &= (( c )) x (( "%+d"|format(d) )) \\
  (( a )) x (( "%+d"|format(-c) ))x &= (( d )) (( "%+d"|format(-b) )) \\
  (( a - c )) x &= (( d - b )) \\
  x &= \frac{(( d - b ))}{(( a - c ))} \\
  x &
        (* if 100*(d-b)/(a-c) == (100*(d-b)/(a-c))|int *)
            =
        (* else *)
            \simeq
        (* endif *)
        (( ((d-b)/(a-c)) | facteur("2") ))
\end{align*}

L'unique solution est
$x
(* if 100*(d-b)/(a-c) == (100*(d-b)/(a-c))|int *)
    =
(* else *)
    \simeq
(* endif *)
(( ((d-b)/(a-c)) | facteur("2") ))$.
