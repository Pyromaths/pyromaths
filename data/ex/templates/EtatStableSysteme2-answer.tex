\exercice*
Puisque $P$ est l'état stable, alors $P=P\times M$.

\begin{align*}
    P\times M &= \begin{pmatrix}x&y\end{pmatrix}\times
\begin{pmatrix}
  \numprint{(( a ))} & \numprint{(( 1-a ))} \\
  \numprint{(( 1-b ))} & \numprint{(( b ))} \\
\end{pmatrix}\\
&= \begin{pmatrix}
  \numprint{(( a ))} x + \numprint{(( 1-b ))} y & \numprint{(( 1-a ))} x + \numprint{(( b ))} y
\end{pmatrix}
\end{align*}

Or $\begin{pmatrix}x&y\end{pmatrix}=P=P\times M$, donc les coefficients des matrices sont deux à deux égaux, donc $x=\numprint{(( a ))} x + \numprint{(( 1-b ))} y$.

D'autre part, puisque $P$ est un état probabiliste, alors $x+y=1$, donc $y=1-x$. Donc, en remplaçant $y$ par $1-x$ dans l'équation précédente, on obtient :

\begin{align*}
  x &= \numprint{(( a ))} x + \numprint{(( 1-b ))} (1-x) \\
  x &= \numprint{(( a ))} x + \numprint{(( 1-b ))} - \numprint{(( 1-b ))}x\\
  x-\numprint{(( a ))} x +\numprint{(( 1-b ))}x &= \numprint{(( 1-b ))}\\
  (1-\numprint{(( a ))}+\numprint{(( 1-b ))})x &= \numprint{(( 1-b ))}\\
  \numprint{(( 2-a-b ))} x &= \numprint{(( 1-b ))}\\
  x &= \frac{\numprint{(( 1-b ))}}{\numprint{(( 2-a-b ))}}\\
x &= \numprint{(( ((1-b)/(2-a-b)) ))}
\end{align*}

Enfin, puisque $y=1-x$, alors $y=1-\numprint{(( ((1-b)/(2-a-b)) ))}=\numprint{(( ((1-a)/(2-a-b)) ))}$.

L'unique état stable de ce graphe est donc $\begin{pmatrix}
\numprint{(( ((1-b)/(2-a-b)) ))} &
\numprint{(( ((1-a)/(2-a-b)) ))}
\end{pmatrix}$.
