\exercice*
Puisque $P$ est l'état stable, alors $P=P\times M$.

\begin{align*}
    P\times M &= \begin{pmatrix}x&y\end{pmatrix}\times
\begin{pmatrix}
(( a | decimal )) & (( (1-a) | decimal )) \\
(( (1-b) | decimal )) & (( b | decimal )) \\
\end{pmatrix}\\
&= \begin{pmatrix}
(( a | decimal )) x + (( (1-b) | decimal )) y & (( (1-a) | decimal )) x + (( b | decimal )) y
\end{pmatrix}
\end{align*}

Or $\begin{pmatrix}x&y\end{pmatrix}=P=P\times M$, donc les coefficients des matrices sont deux à deux égaux, donc $x=(( a | decimal )) x + (( (1-b) | decimal )) y$.

D'autre part, puisque $P$ est un état probabiliste, alors $x+y=1$, donc $y=1-x$. Donc, en remplaçant $y$ par $1-x$ dans l'équation précédente, on obtient :

\begin{align*}
x &= (( a | decimal )) x + (( (1-b) | decimal )) (1-x) \\
x &= (( a | decimal )) x + (( (1-b) | decimal )) - (( (1-b) | decimal ))x\\
x-(( a | decimal )) x +(( (1-b) | decimal ))x &= (( (1-b) | decimal ))\\
(1-(( a | decimal ))+(( (1-b) | decimal )))x &= (( (1-b) | decimal ))\\
(( (2-a-b) | decimal )) x &= (( (1-b) | decimal ))\\
x &= \frac{(( (1-b) | decimal ))}{(( (2-a-b) | decimal ))}\\
x &= (( ((1-b)/(2-a-b)) | decimal ))
\end{align*}

Enfin, puisque $y=1-x$, alors $y=1-(( ((1-b)/(2-a-b)) | decimal ))=(( ((1-a)/(2-a-b)) | decimal ))$.

L'unique état stable de ce graphe est donc $\begin{pmatrix}
(( ((1-b)/(2-a-b)) | decimal )) &
(( ((1-a)/(2-a-b)) | decimal ))
\end{pmatrix}$.
