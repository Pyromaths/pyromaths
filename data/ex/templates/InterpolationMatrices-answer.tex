\begin{enumerate}

  \item
    \begin{enumerate}
      \item
        \begin{itemize}
          \item Puisque $A( (( X[0]|facteur )) ; (( Y[0]|facteur )) )$ est sur la courbe de $f$, alors $f( (( X[0]|facteur )) )=(( Y[0]|facteur ))$, soit
            $a \times (( X[0]|facteur ))^2+b\times (( X[0]|facteur ))+c=(( Y[0]|facteur ))$,
            c'est-à-dire $(( (X[0]**2)|facteur ))a + (( X[0]|facteur ))b + c = (( Y[0]|facteur ))$.
          \item De même, puisque $B( (( X[1]|facteur )) ; (( Y[1]|facteur )) )$ est sur la courbe de $f$, alors $f( (( X[1]|facteur )) )=(( Y[1]|facteur ))$, soit
            $a \times (( X[1]|facteur ))^2+b\times (( X[1]|facteur ))+c=(( Y[1]|facteur ))$,
            c'est-à-dire $(( (X[1]**2)|facteur ))a + (( X[1]|facteur ))b + c = (( Y[1]|facteur ))$.
          \item Enfin, puisque $C( (( X[2]|facteur )) ; (( Y[2]|facteur )) )$ est sur la courbe de $f$, alors $f( (( X[2]|facteur )) )=(( Y[2]|facteur ))$, soit
            $a \times (( X[2]|facteur ))^2+b\times (( X[2]|facteur ))+c=(( Y[2]|facteur ))$,
            c'est-à-dire $(( (X[2]**2)|facteur ))a + (( X[2]|facteur ))b + c = (( Y[2]|facteur ))$.
        \end{itemize}

        On en déduit le système suivant :
        \[ \left\{\begin{array}{rcl}
            (( (X[0]**2)|facteur ))a + (( X[0]|facteur ))b + c &=& (( Y[0]|facteur )) \\
            (( (X[1]**2)|facteur ))a + (( X[1]|facteur ))b + c &=& (( Y[1]|facteur )) \\
            (( (X[2]**2)|facteur ))a + (( X[2]|facteur ))b + c &=& (( Y[2]|facteur )) \\
        \end{array}\right.\]
      \item

\begin{align*}
        \left\{\begin{array}{rcl}
            (( (X[0]**2)|facteur ))a + (( X[0]|facteur ))b + c &=& (( Y[0]|facteur )) \\
            (( (X[1]**2)|facteur ))a + (( X[1]|facteur ))b + c &=& (( Y[1]|facteur )) \\
            (( (X[2]**2)|facteur ))a + (( X[2]|facteur ))b + c &=& (( Y[2]|facteur )) \\
        \end{array}\right.
&\iff
\begin{pmatrix}
(( (X[0]**2)|facteur ))a + (( X[0]|facteur ))b + c \\
(( (X[1]**2)|facteur ))a + (( X[1]|facteur ))b + c \\
(( (X[2]**2)|facteur ))a + (( X[2]|facteur ))b + c \\
\end{pmatrix} = (( Y|zip|matrice ))\\
&\iff
(( M|matrice)) \times (( [["a"], ["b"], ["c"]]|matrice )) = (( Y|zip|matrice )) \\
&\iff M X=R
\end{align*}

        Avec : $M= (( M|matrice ))$, $X= (( [["a"], ["b"], ["c"]]|matrice ))$ et $R= (( Y|zip|matrice ))$.
    \end{enumerate}
  \item
    Comme $M$ est inversible, et que $MX = R$, alors $X = M^{-1}\times R$. À la calculatrice, on obtient
$M^{-1}\times R=(( M|matrice ))^{-1}\times ((Y|zip|matrice)) = (( A|zip|matrice ))$.

    Ainsi, $a=(( A[0]|facteur ))$, $b=(( A[1]|facteur ))$, et $c=(( A[2]|facteur ))$.
  \item
En utilisant les valeurs de $a$, $b$, et $c$ calculées précédemment, nous connaissons l'expression de la fonction : $f(x) = (( A[0]|facteur ))x^2 + (( A[1]|facteur))x + (( A[2]|facteur ))$.

Nous pouvons maintenant calculer l'image de $(( x|facteur ))$ par cette fonction :
\begin{align*}
f( (( x|facteur )) )
&= (( A[0]|facteur )) \times (( x|facteur ))^2 + (( A[1]|facteur )) \times (( x|facteur)) + (( A[2]|facteur ))\\
&= (( (A[0]*x**2+A[1]*x+A[2])|facteur ))
\end{align*}
Donc $f( (( x|facteur )) ) = (( (A[0]*x**2+A[1]*x+A[2])|facteur ))$.
\end{enumerate}

